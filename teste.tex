%%%%%%%%%%%%%%%%%%%%%%%%%%%%% Define Article %%%%%%%%%%%%%%%%%%%%%%%%%%%%%%%%%%
\documentclass[twocolumn]{article}
%%%%%%%%%%%%%%%%%%%%%%%%%%%%%%%%%%%%%%%%%%%%%%%%%%%%%%%%%%%%%%%%%%%%%%%%%%%%%%%

%%%%%%%%%%%%%%%%%%%%%%%%%%%%% Using Packages %%%%%%%%%%%%%%%%%%%%%%%%%%%%%%%%%%
\usepackage{geometry}
\usepackage{graphicx}
\usepackage{amssymb}
\usepackage{amsmath}
\usepackage{amsthm}
\usepackage{empheq}
\usepackage{algpseudocode}
\usepackage{algorithm}
\usepackage{mdframed}
\usepackage{booktabs}
\usepackage{lipsum}
\usepackage{graphicx}
\usepackage{tikz}
\usepackage{color}
\usepackage{psfrag}
\usepackage{bm}
\usepackage[brazil]{babel}
\usepackage{babelbib}
%%%%%%%%%%%%%%%%%%%%%%%%%%%%%%%%%%%%%%%%%%%%%%%%%%%%%%%%%%%%%%%%%%%%%%%%%%%%%%%

% Other Settings

%%%%%%%%%%%%%%%%%%%%%%%%%% Page Setting %%%%%%%%%%%%%%%%%%%%%%%%%%%%%%%%%%%%%%%
\geometry{a4paper}

%%%%%%%%%%%%%%%%%%%%%%%%%% Define some useful colors %%%%%%%%%%%%%%%%%%%%%%%%%%
\definecolor{ocre}{RGB}{243,102,25}
\definecolor{mygray}{RGB}{243,243,244}
\definecolor{deepGreen}{RGB}{26,111,0}
\definecolor{shallowGreen}{RGB}{235,255,255}
\definecolor{deepBlue}{RGB}{61,124,222}
\definecolor{shallowBlue}{RGB}{235,249,255}
%%%%%%%%%%%%%%%%%%%%%%%%%%%%%%%%%%%%%%%%%%%%%%%%%%%%%%%%%%%%%%%%%%%%%%%%%%%%%%%

%%%%%%%%%%%%%%%%%%%%%%%%%% Define an orangebox command %%%%%%%%%%%%%%%%%%%%%%%%
\newcommand\orangebox[1]{\fcolorbox{ocre}{mygray}{\hspace{1em}#1\hspace{1em}}}
%%%%%%%%%%%%%%%%%%%%%%%%%%%%%%%%%%%%%%%%%%%%%%%%%%%%%%%%%%%%%%%%%%%%%%%%%%%%%%%

%%%%%%%%%%%%%%%%%%%%%%%%%%%% English Environments %%%%%%%%%%%%%%%%%%%%%%%%%%%%%
\newtheoremstyle{mytheoremstyle}{3pt}{3pt}{\normalfont}{0cm}{\rmfamily\bfseries}{}{1em}{{\color{black}\thmname{#1}~\thmnumber{#2}}\thmnote{,--,#3}}
\newtheoremstyle{myproblemstyle}{3pt}{3pt}{\normalfont}{0cm}{\rmfamily\bfseries}{}{1em}{{\color{black}\thmname{#1}~\thmnumber{#2}}\thmnote{,--,#3}}
\theoremstyle{mytheoremstyle}
\newmdtheoremenv[linewidth=1pt,backgroundcolor=shallowGreen,linecolor=deepGreen,leftmargin=0pt,innerleftmargin=20pt,innerrightmargin=20pt,]{theorem}{Theorem}[section]
\theoremstyle{mytheoremstyle}
\newmdtheoremenv[linewidth=1pt,backgroundcolor=shallowBlue,linecolor=deepBlue,leftmargin=0pt,innerleftmargin=20pt,innerrightmargin=20pt,]{definition}{Definição}[section]
\theoremstyle{myproblemstyle}
\newmdtheoremenv[linecolor=black,leftmargin=0pt,innerleftmargin=10pt,innerrightmargin=10pt,]{problem}{Problem}[section]
%%%%%%%%%%%%%%%%%%%%%%%%%%%%%%%%%%%%%%%%%%%%%%%%%%%%%%%%%%%%%%%%%%%%%%%%%%%%%%%


%%%%%%%%%%%%%%%%%%%%%%%%%%%%%%% Title & Author %%%%%%%%%%%%%%%%%%%%%%%%%%%%%%%%
\title{Resumo Heurísticas Construtivas}
\author{Pedro Tavares de Carvalho}
%%%%%%%%%%%%%%%%%%%%%%%%%%%%%%%%%%%%%%%%%%%%%%%%%%%%%%%%%%%%%%%%%%%%%%%%%%%%%%%

\begin{document}
    \maketitle
    \abstract{
        Nesse documento resumirei os capítulo 15, 16 e 35 do
        Cormen ~\cite{cormen} além do capítulo 8 de Arts\cite{arts}.
    }
    \section{Capítulo 15} % (fold)

    \label{sec:Capítulo 15}
        Nesse capítulo exploraremos um pouco sobre programação dinâmica, descorrendo sobre alguns algoritmos,
        incluindo \emph{Rod Cutting} e multiplicações de matrizes em cadeia. No final, também discutimos sobre
        as partes principais de programação dinâmica e seus conceitos mais explorados.

        A principal caracterização do uso comum programação dinâmica é a do compartilhamento de subproblemas ao se dividir
        o problema principal, o que normalmente se gera repetição de trabalho, que pode ser minimizado ao se utilizar desse método,
        que consiste, em base, em:

        \begin{enumerate}
            \item Caracterizar a estrutura de uma solução ótima
            \item Definir uma solução ótima recursivamente
            \item Computar o valor de uma solução ótima, geralmente de forma \emph{bottom up}
            \item Contruir uma solução ótima a partir da informação computada anteriormente
        \end{enumerate}

        \subsection{15.1} % (fold)
        \label{sub:15.1}
            Nessa sessão, exploraremos o conceito do problema de \emph{rod cutting}.
            \begin{definition}
                Problema \emph{Rod Cutting}

                A partir de uma barra de metal de tamanho $r$, podemos fazer cortes discretos na mesma.
                Cada tamanho de barra é vendido por um valor diferente conhecido. O objetivo do problema é
                maximizar o valor total da barra a partir dos cortes feitos nela.

            \end{definition}

            O problema pode ser resolvido de forma recursiva, se imaginarmos cada corte como dois subproblemas,
            onde temos que maximizar os cortes na barra da esqueda e na barra da direita. Um algoritmo recursivo
            para resolver esse problema é escrito a seguir.
            \begin{algorithm}

            \end{algorithm}
            % subsection 15.1 (end)

        % section Capítulo 15 (end)




    \bibliography{bib.bib}{}
    \bibliographystyle{babunsrt}
\end{document}
